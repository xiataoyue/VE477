\documentclass[12pt, a4paper]{article}
\usepackage{enumerate}
\usepackage{amsmath}
\usepackage{amssymb}
\usepackage{blkarray}
\usepackage{geometry}
\usepackage{float}
\usepackage{graphicx}
\usepackage[linesnumbered, ruled]{algorithm2e}
\usepackage{tikz}
\usepackage{diagbox}

\geometry{left = 2.0cm, right = 2.0cm}

\SetKwInOut{Input}{Input}
\SetKwInOut{Output}{Output}
\SetKwProg{Fn}{Function}{:}{end}

\title{
    \begin{figure}[H]
        \centering
        \includegraphics[width=7cm, height=5cm]{AAA.png}
    \end{figure}
    VE477 Introduction to Algorithms\\ 
    Homework 8}
\author{Taoyue Xia, 518370910087}
\date{2021/12/06}

\begin{document}
\maketitle

\newpage

\section*{Ex. 1 --- Fast multi-point evaluation and interpolation}
\begin{center}
    \textit{Part I --- Fast multi-point evaluation}
\end{center}
\begin{enumerate}
    \item The binary tree is as follows,
    
          \begin{center}
              \begin{tikzpicture}
                \node{$U$}
                child{ node {$u_0, \dots, u_{n/2-1}$}
                    child{ node {$u_0,\dots,u_{n/4-1}$}
                        child{ node {$\dots$} }
                        child{ node {$\dots$} }
                    }
                    child[missing]{}
                    child{ node {$u_{n/4},\dots,u_{n/2-1}$}
                        child{ node {$\dots$} }
                        child{ node {$\dots$} }
                    }
                }
                child[missing]{}
                child[missing]{}
                child[missing]{}
                child{ node {$u_{n/2}, \dots, u_{n-1}$}
                    child{ node {$u_{n/2},\dots,u_{3n/4-1}$}
                        child{ node {$\dots$} }
                        child{ node {$\dots$} }
                    }
                    child[missing]{}
                    child{ node {$u_{3n/4},\dots,u_{n-1}$}
                        child{ node {$\dots$} }
                        child{ node {$\dots$} }
                    }
                };
              \end{tikzpicture}
          \end{center}
    \item Firstly, when $i = 0$, we can have
          \[M_{0, j} = \prod_{l=0}^{2^0 - 1}m_{j2^0+l} = m_{j+0} = m_{j}\]
          Then for general cases,
          \begin{align*}
              M_{i+1, j} &= \prod_{l=0}^{2^{i+1}-1}m_{j2^{i+1}+l}\\
                         &= \prod_{l=0}^{2^i-1}m_{j2^{i+1}+l} \cdot \prod_{l=2^i}^{2^{i+1}-1}m_{j2^{i+1}+l}\\
                         &= \prod_{l=0}^{2^i-1}m_{2j\cdot 2^i+l}\cdot \prod_{l=0}^{2^i-1}m_{2j\cdot 2^i + l + 2^i}\\
                         &= \prod_{l=0}^{2^i-1}m_{(2j)\cdot 2^i+l}\cdot \prod_{l=0}^{2^i-1}m_{(2j+1)\cdot 2^i + l}\\
                         &= M_{i, 2j}\cdot M_{i, 2j+1}
          \end{align*}
          With the above two deduction, we have proved that,
          \[
          \begin{cases}
              M_{i+1, j} &= M_{i, 2j}M_{i, 2j+1}\\
              M_{0, j}   &= m_j
          \end{cases}
          \]
    \item $M_{0, j}$ can be seen as the $n$ leaves $m_0,\dots,m_{n-1}$, and $M_{i, j}$ is the $j_{th}$ elements at depth $i$. 
          Also, For $M_{i+1, j}$, it has two children $M_{i, 2j}$ and $M_{i, 2j+1}$. Therefore, as the depth increases by 1, 
          the number of nodes will shrink to half of the lower level nodes. So at the final depth $k$, 
          the $n$-degree polynomial $M_{k, 0}$ would exist.
    \item \begin{enumerate}[a)]
        \item The algorithm is shown below,
              
              \begin{algorithm}[!htb]
                  \caption{Subproduct Tree Build}
                  \Input{$n = 2^k$ a power of 2, $U = \{u_0, \dots, u_{n-1}\}$}
                  \Output{$M_{k, 0}$}
                  \BlankLine
                  \For{$j = 0$ to $n-1$}{
                      $M_{0, j} \leftarrow x - u_j$\;
                  }
                  \For{$i = 1$ to $k$}{
                      \For{$j = 0$ to $2^{k-i} - 1$}{
                          $M_{i, j} \leftarrow M_{i-1, 2j}M_{i-1, 2j+1}$\;
                      }
                  }
                  \KwRet{$M_{k, 0}$}\;
              \end{algorithm}
        \item The algorithm is shown below,
              
              \begin{algorithm}[!htb]
                  \caption{Recursive Subproduct Tree Downward}
                  \Input{$n = 2^k$, $P$ a polynomial of degree less than $n$, set $\{u_0, \dots, u_{n-1}\}$, $M_{i, j}$}
                  \Output{$P(u_0), \dots, P(u_{n-1})$}
                  \SetKwFunction{rd}{RecurseDown}
                  \BlankLine
                %   \tcc{All the $n$ $P$-values will be saved in an array}
                  \Fn{\rd{$arr$, $M_{i, j}$}, $P$}{
                      \If{$i == 0$}{
                          $arr[2j] \leftarrow P$\;
                          \KwRet\;
                      }
                      \rd{$arr$, $M_{i-1, 2j}$, $P \mod M_{i-1, 2j}$}\;
                      \rd{$arr$, $M_{i-1, 2j}$, $P \mod M_{i-1, 2j+1}$}\;
                  }
                  $arr \leftarrow$ an empty array\;
                  \rd{$arr$, $M_{k, 0}$, $P$}\;
                  \KwRet{$arr$}\;
                  \tcc{$arr$ contains $P_{u_0}, \dots, P(u_{n-1})$}
              \end{algorithm}
    \end{enumerate}
    \item \begin{enumerate}[a)]
        \item When $k = 0$, then the polynomial $M$ has degree 1, then $P$ can only be degree 0, which can be returned directly.
          
            Suppose that for $k = m\ (m \in \mathbb{N}^*)$, it returns $P(u_0), \dots, P(u_{2^m - 1})$. Then for $k = m + 1$, 
            at level $m$, all the $M_{1, j}$ has degree 2, so it is possible for $P$ at that level to have degree 1, 
            which does not give an answer. Therefore, we need to go down another level to $m+1$, 
            so that we can have the values $P(u_0, \dots, P(u_{2^{m+1} - 1}))$. Proof done.
        \item From the above algorithm, we can write the time needed in every level,
              \[T(n) = 2T(n/2) + \mathcal{O}(M(n))\]
              Therefore, according to the Master's Theory, the overall time complexity is $\mathcal{O}(M(n)\log n)$.
    \end{enumerate} 
\end{enumerate}

\newpage
\begin{center}
    \textit{Part II --- Fast interpolation}
\end{center}

\begin{enumerate}
    \item Knowing the points $(x_i, y_i)$, the Lagrange Interpolation method can be expressed as:
          \[
            \ell_i(x) = \prod_{\begin{smallmatrix} 0\leq m\leq k\\ m \neq i \end{smallmatrix}} \frac{x - x_m}{x_i - x_m} = 
            \frac{(x - x_0)\cdots (x - x_{i-1})(x - x_{i+1})\cdots (x - x_k)}{(x_i - x_0)\cdots (x_i - x_{i-1})(x_i - x_{i+1})\cdots (x_i - x_k)}
          \]
          \[L(x) = \sum_{i=0}^k y_i \ell_i(x)\]
          For our question, having $u_0, \dots, u_{n-1}$ as $x_i$ and $P(u_0), \dots, P(u_{n-1})$ as $y_i$, 
          and $m = \prod_{i=0}^{n-1}(X - u_i)$, we can have
          \[P = \sum_{i=0}^{n-1}(y_i\cdot \prod_{j\neq i}\frac{1}{u_i - u_j}\cdot \frac{m}{X - u_i})\]
          Moreover, for each $u_i - u_j$ and $i \neq j$, we can find a $v_{i, j}$ in $R$ such that $(u_i - u_j)v_{i, j} = e$, 
          the unit element.
    \item Knowing that $m = \prod_{i=0}^{n-1}(x - u_i)$, denote $p_i = \prod_{j=0}^{i-1}(x - u_j)$ we can deduce that,
          \[m^\prime = (\prod_{i=0}^{n-1}(x - u_i))^\prime = p_{n-1} + (x - u_{n-1})p_{n-1}^\prime\]
          \[p_{n-1}^\prime = p_{n-2} + (x - u_{n-2})p_{n-2}^\prime\]
          Then we can iterate this progress until it reaches $p_1$, and by expansion, the expression is 
          \[m^\prime = \frac{m}{x - u_{n-1}} + \frac{m}{x - u_{n-2}} + \dots + \frac{m}{x - u_0} = \sum_{i=0}^{n-1}\frac{m}{x - u_i}\]
          According to this equation, we can have
          \[m^\prime(u_i) = \prod_{j\neq i}(u_i - u_j) = \frac{1}{s_i}\]
          Since all the terms containing $x - u_i$ is 0.
          Proof done.
    \item The algorithm is as follows,
    
          \begin{algorithm}[!htb]
              \caption{Interpolation for $P$}
              \Input{$n = 2^k$, $u_0, \dots, u_{n-1}$, $M_{i, j}$, $P_{u_0}, \dots, P(u_{n-1})$}
              \Output{$P$}
              \SetKwFunction{IP}{Interpolation}
              \BlankLine
              \Fn{\IP{$P(u_i)$, $M_{i, j}$}}{
                  \If{$i = 0$} {
                      \KwRet{$P(u_j)$}\;
                  }
                  $l \leftarrow$ \IP{$P(u_i)$, $M_{i-1, 2j}$}\;
                  $r \leftarrow$ \IP{$P(u_i)$, $M_{i-1, 2j+1}$}\;
                  result $\leftarrow l\cdot M_{i-1, 2j} + r\cdot M_{i-1, 2j+1}$\;
                  \KwRet{result}\;
              }
          \end{algorithm}
          \newpage
    \item Correctness and complexity
          \begin{enumerate}[a)]
              \item Same induction as in Part I 5(a).
              \item For the deduction we have discussed in question 2, the time complexity of computing $m^\prime$ can be expressed as
                    \[T(n) = 2T(n/2) + M(n)\]
                    After calculating $m^\prime$, we just need to insert $u_i$ and take the reciprocal, we can get $s_i$. 
                    Therefore, we can prove that computing $s_i$ in question 2 amounts to $\mathcal{O}(M(n)\log n)$ operations in $R$.
              \item The time complexity for construct the subproduct tree is $\mathcal{O}(M(n)\log n)$, 
                    and the time complexity of calculation for $s_i$ is also $\mathcal{O}(M(n)\log n)$. 
                    After combining these two, we can perform the interpolation in linear time. 
                    Therefore, the interpolation problem can be solved in $\mathcal{O}(M(n)\log n)$ ring operations.
          \end{enumerate}
    \item The possibility depends on the memory space needed for storing $M_{i, j}$.

\end{enumerate}

\newpage
\section*{Ex. 2 --- Critical thinking}
\begin{enumerate}
    \item If $xy = e$, then $y = x^{-1}$, so $xy = yx = e$.
          
          If $xy \neq e$, replace $y$ with $x^{-1}y$ (since $\forall x, y\in G$, $x^{-1}y$ can form a bijection $G \rightarrow G$). 
          Then we can have 
          \begin{align*}
              (xx^{-1}y)^2 &= (x^{-1}yx)^2\\
              y^2          &= (x^{-1}yx)(x^{-1}yx)\\
              y^2          &= x^{-1}y^2x\\
              xy^2         &= yx^2
          \end{align*}
          Since it is mentioned in the stem that $\forall x \neq e,\ x^2 \neq e$, we can reduce $y^2$ and $x^2$ to $y$ and $x$, 
          while keeping generality. Therefore, we can have $xy = yx$, which proves that $G$ is abelian.
    \item \begin{itemize}
        \item Since $S_2$ says that he cannot infer the two numbers, it means that the two numbers are not both prime, 
              because only when the number can be factorized as two prime numbers can $S_2$ figure them out.
        \item Then $S_1$ says that he knew $S_2$ cannot know the two numbers, 
              it means that his $x+y$ are not any of the sum of two primes. Then all the even numbers can be ignored, 
              according to the Goldbach's conjecture, 
              all the even numbers smaller than $4\times 10^18$ can be represented as the sum of two primes. 
              Also, the sum of $2$ and other odd primes can be ignored too. After the extraction, 
              we only have the numbers $x+y = 11, 17, 23, 27, 29, 35, 37, 41, 47, 53$ remaining.
        \item Then according to what $S_1$ has said, $S_2$ would know that $x+y$ must fall in these numbers, 
              and the product $xy$ can only refer to one $x + y$.
        \item Finally, we can know that the sum is $17$ and $x = 4, y = 13$ or $x = 13, y = 4$.
    \end{itemize}

\newpage
\section*{Ex. 3 --- Beyond ve477}
Swype was a virtual keyboard for touchscreen smartphones and tablets, 
where the user enters words by sliding a finger or stylus from the first letter of a word to its last letter, 
lifting only between words.

It could be implemented in the following steps:
\begin{itemize}
    \item Firstly, we need to trace the path of user's finger, when it detects an obvious angle, 
          the letter at the angle could be recorded.
    \item Then it comes to the priority of the display sequence of words, from common to rare.
    \item After that, we can record users' priority and habits to offer personalized word association.
    \item Finally, it comes to the important function, the error correction. When it detects some strange word, 
          which may be caused by mistyping, it can correct it probably using AI or other algorithms, and hint for the right one.
\end{itemize}
          
\end{enumerate}

\end{document}